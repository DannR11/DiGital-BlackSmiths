\documentclass[10pt]{article}
\usepackage{sectsty}
\subsectionfont{\small}


\begin{document}

\begin{titlepage}
	\centering
	\vspace*{\fill}
	
	\vspace*{0.5cm}
	
	\huge\bfseries
	\rule{\textwidth}{1.6pt}\\[\baselineskip]
	Thutong Site Learning Center \\Outline of Requirements and Design Documentation
	
	\vspace*{0.5cm}
	
	\large Edited by: \\[\baselineskip]
	
	{Fiwa lekhulani\\Daniel Rocha\\lebogang Ntatleng\\Lesego Mabe\\Tlou Lebelo\\Oluwatosin Botti}
	
	\rule{\textwidth}{1.6pt}\\[\baselineskip]
	
	
	\vspace*{\fill}
\end{titlepage}

\newpage

\paragraph*{•}
\section*{Non-Functional Requirements} 

Requirements as stipulated by client documentation
The requirement will be categorized according to performance, quality, security and interface requirements.

\subsection*{Performance Requirements}
\begin{itemize}

\item The system must perform efficiently and fast despite the number of users. With that said it must be able to sustain a large number of users as it grows and develops further.
\item It must also be able to work efficiently despite the content linked to or attached to the database. There must be effective content management.{\newline \footnotesize It was also suggested that content be limited to a specific size.}
\item The interaction between the server, database and the actual delivery of content to the user must be speedily and correct
\item System must be easy to expand
\end{itemize}

\subsection*{Quality Requirements}
\begin{itemize}
\item System highly relies on database content {\newline \footnotesize Which will need to be of approved quality from experts. Also managing content to be passed to and fro without breaking or being heavy on the system}
\item Availability covered in context of should be responsive as long as user has sufficient internet access
\end{itemize}

\subsection*{Security Requirements}
\begin{itemize}
\item It is necessary that each user have secure access to their personalised accounts that can only be accessed after being authenticated. (requires a verification process, method to retrieve password if forgotten and make a user aware that their account was tried to be accessed)
\item Parents need to able to verify, access, monitor and/or approve child’s account. This is to see their progress but also manage the internet exposure of the learner.
\item Parents need to able to verify, access, monitor and/or approve child’s account. This is to see their progress but also manage the internet exposure of the learner.
\item The administrators account must be most secure as from this account all aspects of the Learning Centre must be accessible.
\end{itemize}

\subsection*{Interface Requirements}
\begin{itemize}
\item Simplistic layout; not have clusters of information distracting the user
\item Website must be intuitive (easy to learn); also include instructional videos or prompts to facilitate process
\item Website to be implemented using relevant web development technologies
\item Various users should have different layouts upon logging in relevant to the tasks they perform.
\item Website should be accessible from various devices (PC laptop, desktop) and still respond and work accordingly
\end{itemize}

\section*{Functional Requirements}
\subsection*{All Users}
\begin{itemize}
\item all users should be able to register, login to a account
\item all users should be able to view advertisments, job opportunities and bursaries
\item all users should be able to search and view academic and marketing content
\end{itemize}

\subsection*{Student User}
\begin{itemize}
\item Students should be able to connect their account to Facebook
\item Students should be able to search or select lessons of a particular grade and/or subject
\item Students should be able to do a quiz after completing a lesson and receive a mark for that quiz.
\item students should be able to see their progress as to the topics, lessons and quizes they've completed
\end{itemize}

\subsection*{Marketing Consultant}
\begin{itemize}
\item Marketing consultants should be able to able to add advertising content either in the form of jobs, events, advertisments, notices/ notifications as well as bursaries and internships.
\end{itemize}

\subsection*{Expert Consultant}
\begin{itemize}
\item An expert consultant should be able to upload notes, videos or links of the latter to create lessons
\item An expert consultant should be able to be able to  remove any of the content they have uploaded
\item An expert consultant should be able to be able to create quizes and it's marking criteria
\end{itemize}

\subsection*{Administrator}
\begin{itemize}
\item An administrator should be able to view all profiles of all users, from there they should be able to remove content or an account they feel is not suitable to the system.
\item An administrator can perform the same actions as both the marketing and expert cpnsultants
\end{itemize}

\section*{Architectural System}
\subsection*{Introduction}
We have chosen to work with a 3 Tier system with the layers of that system being firstly, the User Interface Layer, secondly the Server Layer and thirdly the database layer.

\subsection*{Presentation Layer}
The UI layer is our “presentation layer.” It organizes what each kind of user will see and have available for interaction. It will be responsible for handling the presentation in a format that is suitable for each one of the user types in our system. Technologies used to implement this layer will include: HTML5, CSS3, JavaScript, Ajax, Bootstrap, jQuery and PHP. The UI layer and the server layer will be communicating using the HTTP protocol.
\subsection*{Server Layer}
The Server layer is where have our business logic and processing and coordination of the system. It will handle interfacing between the user interface layer and the data layer as well as handling advertising logic as well, and technologies used to implement this layer include: PHP, Apache HTTP Server. The Server layer will be communicating with the Database Layer through PHP's mysqli connector. 
\subsection*{Database Layer}
The Database layer will be working hand in hand with the Server layer as the Server layer will be supplying it with data for the UI layer (inputted by the users) and will also obtain data from the database layer as required by user requests (queries). The data layer will house a database implemented in MySQL and communicating through PHP's mysqli connector. 


\end{document}

